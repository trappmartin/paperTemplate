\documentclass{article}
\usepackage{arxiv}

% load shortex containing useful packages and shorthand notation
\usepackage{shortex}
\usepackage{safecolours} % use colour-blind friendly colours

\usepackage{natbib}
\bibliographystyle{abbrvnat}

\usepackage[utf8]{inputenc} % allow utf-8 input
\usepackage[T1]{fontenc}    % use 8-bit T1 fonts
\usepackage{url}            % simple URL typesetting
\usepackage{booktabs}       % professional-quality tables

% Latin
\usepackage{xspace}
\newcommand{\eg}{\textit{e.g.}\xspace}
\newcommand{\ie}{\textit{i.e.}\xspace}
\newcommand{\cf}{\textit{cf.}\xspace}
\newcommand{\wrt}{\textit{w.r.t.}\xspace}

% useful macros
\def\todo#1{\textcolor{red}{\textsc{#1}}\xspace}
\def\note#1{\textcolor{blue}{\textit{#1}}\xspace}

% nicer epsilon
\newcommand{\oldepsilon}{\epsilon}
\renewcommand{\epsilon}{\varepsilon}

% Fill out paper:
\begin{document}
\title{Fancy Paper Title}

\author{%
	Author A \and Author B
}

\maketitle

\begin{abstract}
    Add abstract here..
\end{abstract}


\section{Project Plan}\label{sec:plan}

\note{The following outline can help in planning a research project:}

\subsection{Problem Setting (P1)}
\todo{What is the problem that is addressed, and why is it relevant?}

\subsection{Existing Approaches (S1)}
\todo{How has this been solved/approached so far?}

\subsection{Limitations (P2)}
\todo{What are the problems with related work, and why do they fail to address the problem?}

\subsection{Your Proposal/Solution}
\todo{Describe the overall project proposal idea and how this aims to solve P1 and P2 and links to S1.}

\subsection{Key Contributions}
\todo{What are the key contributions you expect?}

\subsection{Metrics}
\todo{How can you measure if you succeeded, for example, what metrics to use?}

\subsection{Datasets}
\todo{Are there benchmarks? What data can be used? Do you need to collect data?}

\subsection{Impact}
\todo{What are practical contributions that you expect, and what are relevant use cases?}

\clearpage

\section{Introduction}\label{sec:introduction}
\note{Introduction section of your work, including motivation and relation to existing work (P1, S1, \& P2).}

\section{Background}\label{sec:background}
\note{Technical background (optional).}

\section{Main}\label{sec:main}
\note{Description and technical details of your work.}

\section{Experiments}\label{sec:experiments}
\note{Description and discussion of experiments, experimental results, and ablations.}

\section{Conclusion and Discussion}\label{s:conclusion}
\note{Concluding remarks and discussion of your work and results.}

\subsection{Limitations}
\note{Discussion of the limitations of your work.}


\subsection*{Acknowledgements}
\subsection*{Author Contributions}

% print bibliography
\bibliography{bibliography}
\nocite{*}

% Fill out appendix:
\newpage
\appendix
\section{Appendix title}\label{a:appendix1}


\end{document}
